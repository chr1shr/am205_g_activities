\documentclass{article}
\usepackage[utf8]{inputenc}
\usepackage{abstract}
\usepackage{ marvosym }
\usepackage[margin=1in]{geometry}
\usepackage{amssymb }
\usepackage{bm}
\usepackage{amsmath,amsthm}
\usepackage{graphicx}
\usepackage{subcaption}
\usepackage{listings}
\usepackage{bm}
\usepackage{dsfont }
\usepackage{xcolor}
\usepackage{hyperref}
\usepackage{algpseudocode}
\usepackage{listings}
\usepackage{stackengine}
\usepackage{soul}

\hypersetup{
   colorlinks=true, linktocpage=true, pdfstartpage=3, pdfstartview=FitV,
   breaklinks=true, pdfpagemode=UseNone, pageanchor=true, pdfpagemode=UseOutlines,
   plainpages=false, bookmarksnumbered, bookmarksopen=true, bookmarksopenlevel=1,
   hypertexnames=true, pdfhighlight=/O,
   urlcolor=blue, linkcolor=blue, citecolor=green,
   pdfauthor={Michael S. Emanuel},
   pdfsubject={Harvard AM205 (Fall 2021)},
   pdfkeywords={},
   pdfcreator={pdfLaTeX},
   pdfproducer={LaTeX with hyperref}
}

% Original commands
\newcommand{\nnn}{\\ \nonumber \\ \nonumber}

% MSE Misc
\newcommand{\Z}{\mathbb{Z}}
\newcommand{\half}{\frac{1}{2}}
\newcommand{\grad}{\nabla}
\newcommand{\loss}{\mathcal{L}}
\newcommand{\sign}{\text{sign}}
\newcommand{\nn}{\nonumber \\}

% MSE statistics
\newcommand{\E}{\mathrm{E}}
\newcommand{\var}{\mathrm{Var}}
\newcommand{\cov}{\mathrm{Cov}}
\newcommand{\N}{\mathcal{N}}
\newcommand{\pois}{\mathrm{Pois}}
\newcommand{\thetavec}{\bm{\theta}}
\newcommand{\thetavechat}{\hat{\bm{\theta}}}
\newcommand{\phihat}{\hat{\phi}}
\newcommand{\rr}{\bm{r}}
\newcommand{\FIX}{\mathcal{I}_{\mathbf{X}}}

% Block matrices
\newcommand*{\vertbar}{\rule[-1ex]{0.5pt}{2.5ex}}
\newcommand*{\horzbar}{\rule[.5ex]{2.5ex}{0.5pt}}

% MSE terminal
\newcommand{\tty}[1]{\texttt{#1}}

\title{Applied Math 205: Integrating the Solar System - Exercise}
\author{Michael S. Emanuel}
\date{Assigned: 29-Oct-2021}
\graphicspath{{figs/}}

\begin{document}
\maketitle

\noindent
\section*{Near Approach of Asteroid Apophis}

The asteroid \href{https://en.wikipedia.org/wiki/99942_Apophis}{Apophis} was discovered in 2004.  
I chose to model it for an AM 225 project because it achieved the highest ever rating 
on the \href{https://en.wikipedia.org/wiki/Torino_scale}{Torino Scale}, which is used
to communicate impact hazard of near-earth objects (NEOs) to the public.\footnote{
The \href{https://en.wikipedia.org/wiki/Palermo_Technical_Impact_Hazard_Scale}{Palermo Scale}
is a logarithmic risk scale that is better suited to scientific work.  
It compares the modeled probability of an asteroid strike over the simulation period to the 
estimated probability of a strike by an object of the same or greater energy from a background
distribution of asteroid strikes.}
Apophis was assigned a rating of 4 on the Torino scale in December 2004,
when the estimated risk of a strike peaked at 2.7\% (approximately 1 in 37).
Apophis was nominally named after the Egyptian god 
\href{https://en.wikipedia.org/wiki/Apep}{Apep}, a.k.a Apophis, 
the god of chaos and darkness and opponent of the sun god Ra.
However it was ``really'' named after a character on the Sci-Fi TV show 
\href{https://en.wikipedia.org/wiki/Stargate_SG-1}{Stargate SG-1}, 
which includes in its fan base the astonomers who discovered it
and thus got the privilege of naming it under IAU rules.

The NASA Sentry system estimates that if Apophis were to strike 
the earth, it would release approximately 750 megatons of energy.\footnote{
A megaton is the energy released by one million tons of TNT.  
Though hardly an SI unit, it has been the customary unit for communicating the destructive energy
of nuclear weapons, which is why it has been adopted for communicating asteroid hazards.}
To put this in context, the largest ever hydrogen bomb tested released 57 megatons,
while the eruption of Krakatoa released 200 megatons.
A strike by Apophis would unleash regional devastation in an area on the order of 
perhaps thousands of square miles, but would not pose a major hazard to the entire planet.
This also gets to the point of planetary defense and close monitoring of asteroids.
Perhaps you may share a healthy skepticism that humanity could replicate the plot of 
Armageddon and successfully deflect an object with a mass of $6.1\cdot 10^{10}$ kg.
Given that the U.S. still doesn't have a working ballistic missile defense system,
this skepticism would be well placed.  
A 10 year plan to evacuate a radius of say 50 miles around a projected impact site,
on the other hand, would be an eminently feasible undertaking.\footnote{
Here is an interesting and not very fun thought experiment.
If NASA calculations showed that Cambridge, MA would be the epicenter 
an Apophis strike in 2029, would you sell your house and move?
Would SEAS relocate not just to Allston but to, say, New York?
I would probably want to run the simulation myself at least once before listing my house!}

\subsection*{Simulating the Current Trajectory of Apophis}
Using the techniques we learned in the Integrating the Solar System seminar, 
simulate the trajectory of Apophis using the most current estimate of its state vectors available from JPL.
Initialize a simulation with the Sun, Earth, Moon, and other planet barycenters 
as of the epoch MJD 59516.
You can either use the built-in Horizons API in Rebound (quickest when it works),
or look up the orbital elements directly on the Web API and then key them in by hand.

Simulate the trajectory of the massive bodies plus Apophis for a time span of 20 years.
This will include near approaches it will make to Earth in 2029 and 2037.
Use a granularity of at most 15 minutes.
Calculate the time of nearest approach and compare your results to any published results you can find online.
If you use some cleverness to interpolate the paths, you can significantly refine your forecast.
However, it may be easier to just set a tighter resolution.\footnote{
Remember, you don't need to save a whole array with all the high resolution position data, 
you just need to find the time and distance of nearest approach.
Given the speed and memory capacity of modern computers, I suggest using brute force.}
Report your results as follows:
\begin{itemize}
\item Closest distance of approach in AU (thousands of km)
\item Time of closest approach as an MJD (Date and Time in UTC)
\item Plot of the distance from Earth to Apophis vs. time, with y-axis on a log scale
\end{itemize}

\subsection*{Simulating 1000 Sampled Trajectories of Apophis}
One interesting technical aspect of this problem is the details of combining multiple 
observations into a full uncertainty estimate for the distribution of possible paths.\footnote{
The Sentry system provides estimates of the values and uncertainties of the
orbital elements that can be accessed at the JPL Small-Body browser.
Like the Horizons system, this is an excellent resource that is free to the public.
CNEOS supplies not only current estimates and uncertainties for 
the orbital elements, but a full 6x6 covariance matrix parameterized in the following six 
orbital elements:
\begin{itemize}
\item $a$, the semi-major axis; named \tty{a} in JPL and REBOUND
\item $e$, the eccentricity; named \tty{e} in both systems
\item $i$, the inclination; named \tty{i} in JPL and \tty{inc} in REBOUND
\item $\Omega$, the longitude of the ascending node; named \tty{node} in JPL and \tty{Omega} in REBOUND
\item $\omega$, the argument of perihelion; named \tty{peri} in JPL and \tty{omega} in REBOUND
\item $t_p$, the time of perihelion passage; named \tty{tp} in JPL and \tty{T} in REBOUND
\end{itemize}
Distances are in A.U. in both JPL and REBOUND.  
Angles are quoted in degrees in JPL and in radians in REBOUND.
The time $t_p$ is measured in Julian Days at JPL.
}
However, this is overkill for a short assignment.
I only mention it here because it's interesting.

For this exercise, assume that there is no uncertainty in the absolute position of Apophis,
and that there is a relative uncertainty of $1.0 \times 10^{-6}$ in the velocity of Apophis.
By this I mean that the uncertainty in its velocity vector is normally distributed in three dimensions
with standard deviation $1.0 \times 10^{-6} \cdot \|v\|$, 
where $\|v\|$ is the speed of the currently estimated velocity.

Randomly simulate 1000 trajectories of Apophis by perturbing the velocity 
according to the specifications above.
\begin{itemize}
\item What is the nearest approach to Earth out the 1000 trials?
\item What is the standard deviation of the nearest approach distance?
\end{itemize}

\subsection*{Instructions:}
Please make submit a short PDF report as the main work to be reviewed.\\
Please also supply any Python / Jupyter Notebook files separately on Canvas.

\end{document}
